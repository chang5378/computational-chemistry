\documentclass[11pt]{article}
\usepackage[left=1in, right=1in, top=1in, bottom=1in, nohead]{geometry} % see geometry.pdf on how to lay out the page. There's lots.
\geometry{letterpaper} % or letter or a5paper or ... etc
% \geometry{landscape} % rotated page geometry
\usepackage{outline}
\usepackage{amsmath}
% See the ``Article customise'' template for come common customisations
\usepackage{graphicx}
\usepackage{epstopdf}
\usepackage[version=3]{mhchem}
\usepackage{fancyhdr}
\usepackage{hyperref}
\usepackage{setspace}
\usepackage[labelfont=bf]{caption}

\setlength{\headheight}{10.2pt}
\setlength{\headsep}{20pt}

\def\dbar{{\mathchar'26\mkern-12mu d}}

\pagestyle{fancy}
\fancyhf{}
\renewcommand{\headrulewidth}{0.5pt}
\renewcommand{\footrulewidth}{0.5pt}
\lfoot{\today}
\cfoot{\copyright\ 2015 W.\ F.\ Schneider}
\rfoot{\thepage}
\lhead{\bf{CBE60547}}
\chead{\bf{Computational Chemistry\\Class Project}}
\rhead{\bf{Spring 2015}}

\title{University of Notre Dame\\Advanced Chemical Engineering Thermodynamics\\(CBE 60553)}
\author{Prof. William F.\ Schneider}
%\date{} % delete this line to display the current date


%%% BEGIN DOCUMENT
\begin{document}
\noindent The purpose of the class project is to give you a chance to creatively apply the knowledge you have gained in the course of the semester about first-principles modeling of chemical systems to a short research problem of your own choosing. I hope it is something you can have fun in selecting and carrying out. The research problem could be of several types:

\begin{itemize}
\item A particular chemical problem that you would like to investigate, perhaps related to your own research interests or research project.
\item A particular computational issue you would like to explore, such as the relative performance of some methods (basis sets, core potentials, \ldots) or algorithms for a particular type of calculation.
\item A particular theoretical issue you would like to address, perhaps elaborating on a topic that we didn’t have time to develop in class (e.g., solvation effects, excited states, relativity, \ldots).
\end{itemize}

The project has three requirements:

\begin{enumerate}
\item \textbf{Due March 16, 2015 (10\%).} Provide in pdf, via email to Prof. Schneider, a brief ($< 1$ page) description of your proposed class project. Include (1) background about the problem area, including any relevant references, (2) the specific research question you propose to address, and (3) the computational plan for answering the research question. You may discuss this with Prof. Schneider or the TA before the due-date. You will receive feedback and suggestions shortly after you turn in the proposal.

\item \textbf{Due April 7, 2015 (10\%).} Provide in pdf, via email to Prof. Schneider, a brief ($\leq 2$ page) summary of preliminary computational results, in particular highlighting any difficulties you
have encountered.

\item \textbf{Due April 28, 2015 (80\%).} Your pdf report will include an approximately six-page write-up of your project and results. The scientific and intellectual completeness is more important than any particular length. Include:

  \begin{enumerate}
  \item (5 pts) \textsc{Cover page with title and name}
  \item (15 pts) \textsc{Introduction} to the problem area and specific question you are addressing,
  \item (10 pts) \textsc{Computational methods} applied (software, methods, basis sets, etc.),
  \item (20 pts) \textsc{Results} of your project, including narrative, Tables and Figures, as appropriate,
  \item (15 pts) \textsc{Discussion} of the outcomes and their implications for the question you posed,
  \item (5 pts) \textsc{Conclusions} summarizing outcomes and suggested future work.
  \item (5 pts) \textsc{References cited}
  \item (5 pts) \textsc{Appendix} including input files from your calculations as well as any other details that
might be helpful in understanding the work.
  \end{enumerate}

\noindent You will be graded on the appropriateness of your question, the thoughtfulness and execution of the approach, and the quality and thoroughness of the report of the results.
\end{enumerate}

\newpage

\noindent \textbf{Sample projects from years past:}

\begin{enumerate}
\item United Atom Force Field for Polyacetylene using Ab Initio Calculations and Thermal Conductivity of Single Chain Polyacetylene

\item Computational Measurement of Isothermal Compressibility for Simple Salts

\item Examination of the Conformations of Acetylcholine and its Agonists

\item Hydrogen Bond in Ionic Liquid, Carbon Dioxide and Water System

\item Ab Initio Catalyst Comparison for Ethylbenzene Synthesis from Alkylation

\item GGA Calculation of Strain Effects on \ce{SrTiO3} Band Structure

\item Ethane Radical Reaction: A Comparison of Methyl Radical Termination and Hydrogen Abstraction

\item Computational Modeling of Uranyl and Neptunyl Structures

\item Computational Studies on some Ansa-Zirconocene Compounds

\item Binding of CO upon a Pt(111) Surface

\item Computational Method Comparison for Quadruple Bonded Metal Complexes

\item Amino Acid Zwitterions: An Investigation into the Modern Computational Methods of Solvation
\end{enumerate}

\end{document}
